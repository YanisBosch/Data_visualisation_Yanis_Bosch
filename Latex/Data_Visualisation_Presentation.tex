\documentclass{beamer}
%\usepackage[margin=3.5cm]{geometry}
%\usepackage[latin1]{inputenc}
%\usepackage[T1]{fontenc}
%\usepackage[english]{babel}
%\usepackage{graphicx}
%\usepackage{amsmath}
%\usepackage{amsthm}
%\usepackage{mathtools}
%\usepackage{listings}
%\usepackage{qtree}
%\newtheorem{thm}{Theorem}
%\theoremstyle{definition}
%\newtheorem{defi}{Definition}
%\newtheorem{nota}{Notation}
%\theoremstyle{remark}
%\newtheorem{rem}{Remark}
%\theoremstyle{proposition}
%\newtheorem{prop}{Proposition}
%\newtheorem{lem}{Lemma}
\setbeamertemplate{caption}[numbered]

\begin{document}
	\begin{frame}
		\title{Analysis of housing prices in Connecticut USA}
		\author{Yanis \textsc{Bosch}}
		\maketitle
	\end{frame}

	\begin{frame}{The dataset}
		\begin{enumerate}
			\item Data: Housing market in Connecticut USA from 2001 to 2020
			\item Relevant columns: 
				\begin{enumerate}
   					\item Transaction date
    					\item Sale amount
    					\item Assessed value
				\end{enumerate}
			\item Original provider: Unknown
			\item Refined by: SANDEEP PANDEY
			\item Dataset size: $19\,789 \times 13$
			\item Link: https://www.kaggle.com/datasets/spandey8312/real-estate-2001-to-2020-state-of-connecticut-usa
		\end{enumerate}
	\end{frame}

	\begin{frame}{First plot idea}
		\begin{figure}[h]
			\centering
			\includegraphics[width = 10cm]{Image1}
		\end{figure}
	\end{frame}
	
	\begin{frame}{A closer look (1)}
		\begin{figure}[h]
			\centering
			\includegraphics[width = 10cm]{Image8}
			\caption{\textit{A look at the 4 transactions with the highest $\frac{\text{Sale amount}}{\text{Assessed value}}$ ratio}}
		\end{figure}
	\end{frame}

	\begin{frame}{556 HOLLISTER C-8}
		\begin{figure}[h]
			\centering
			\includegraphics[width = 10cm]{Image7}
			\caption{\textit{Sale price: $33\,328\,000$\$, Assessed value: $19\,880$\$}}
		\end{figure}
	\end{frame}

	\begin{frame}{A closer look (2)}
		\begin{figure}[h]
			\centering
			\includegraphics[width = 10cm]{Image2}
		\end{figure}
	\end{frame}

	\begin{frame}{Cleaning up the data (1)}
		\begin{figure}[h]
			\centering
			\includegraphics[width = 10cm]{Image3}
			\caption{\textit{Removal of all transactions where $\frac{\text{Sale amount}}{\text{Assessed value}} \notin \left[  \frac{1}{10}, 10\right]$ }}
		\end{figure}
	\end{frame}

	\begin{frame}{Cleaning up the data (2)}
		\begin{figure}[h]
			\centering
			\includegraphics[width = 10cm]{Image4}
		\end{figure}
	\end{frame}

	\begin{frame}{Cleaned up plot}
		\begin{figure}[h]
			\centering
			\includegraphics[width = 10cm]{Image5}
		\end{figure}
	\end{frame}

	\begin{frame}{Stock market crash of 2008}
		\begin{figure}[h]
			\centering
			\includegraphics[width = 10cm]{Image6}
		\end{figure}
	\end{frame}
\end{document}









